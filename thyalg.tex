\documentclass[a4paper, 12pt]{scrartcl}

% Most of the preamble is in here.
\usepackage{ianproject}

% Heading
\title{Assessment Autumn 21/22}
\author{}
\modulename{Theory of Algorithms}
\date{Due: last commit on or before August 21\textsuperscript{st}, 2022}


\begin{document}
  
  \maketitle
  
  \vspace{-12mm}
  \begin{center} \textbf{To submit, please email your repository URL to \url{ian.mcloughlin@atu.ie}}. \end{center}


  \noindent These are the instructions for the assessment of Theory of Algorithms in Autumn 2021/2022.
  The assessment is worth 100\% of the marks for the module.
  Please read the \emph{Using git for assessments}~\cite{usinggit} document on the Moodle page which applies here.
  As always, you must also follow the code of student conduct and the policy on plagiarism~\cite{gmitqaf}.

  \section*{Instructions}
  
  The purpose of this assessment is to ensure that you have achieved the learning outcomes of the module while also providing you with sample work to show prospective employers.
  The overall assessment is split into the three interconnected components as detailed below.
  The percentages beside each heading give the weighting of each of the three components.
  You may assume that each bullet point has an equal weighting within its component.
  Note, however, that the examiners' overall impression of your submission may override the individual weightings where deemed appropriate.

  \subsubsection*{GitHub Repository (20\%)}
  Create a GitHub repository containing two Jupyter notebooks -- these are described further down.
  The repository should contain the following.
  \begin{itemize}
    \item Clear and informative \texttt{README.md} explaining why the repository exists, what is in it, and how to run the notebooks.
    \item Comprehensive commit history, with each commit representing a reasonable unit of work.
  \end{itemize}

  \subsubsection*{Post Correspondence Problem Notebook (40\%)}
  Include in your repository a Jupyter notebook called \texttt{post-correspondence.ipynb} that contains the following.
  \begin{itemize}
    \item History and explanation of undecidable problems.
    \item Definition and explanation of the Post Correspondence Problem and the Bounded Post Correspondence Problem.
    \item Explanation as to why the Post Correspondence Problem is undecidable, but the Bounded Post Correspondence Problem is not.
    \item Python function to solve the Bounded Post Correspondence Problem. The function should take two lists of strings (and a bound $k$) and return \texttt{True} if they correspond, \texttt{False} otherwise.
  \end{itemize}
  
  \subsubsection*{Countdown Numbers Game Notebook (40\%)}
  Include in your repository a Jupyter notebook called \texttt{countdown.ipynb} that contains the following.
  \begin{itemize}
    \item Overview and explanation of the Countdown Numbers Game~\cite{countdownnumbers}.
    \item Python function, written in a functional programming style~\cite{pythonfunctional} to solve a Countdown Numbers Game. The function should take a list of six numbers and a target number and return a method to calculate the target from the numbers, if it exists.
    \item Explanation of the functional~\cite{pythonfunctional} aspects of your code.
    \item An explanation and analysis of whether your above Python function contains any side effects.
  \end{itemize}

  \section*{More information about marking}
    In completing the assessment, you should consider the following four aspects of your work.
    It is important that your submission provides direct evidence of each.
    
    \subsubsection*{Research}
    Evidence of research performed on topic; submission based on referenced literature, particularly academic literature; evidence of understanding the documentation for any software or libraries used.
    \subsubsection*{Development}
    Environment can be set up as described; code works without tweaking and as described; code is efficient, clean, and clear; evidence of consideration of standards and conventions appropriate to code of this kind.
    \subsubsection*{Consistency}
    Evidence of planning and project management; pragmatic attitude to work as evidenced by well-considered commit history; commits are of a reasonable size; consideration of how commit history will be perceived by others.
    \subsubsection*{Documentation}
    Clear documentation of how to create an environment in which any code will run, how to prepare the code for running, how to run the code, and what to expect upon running the code. Concise descriptions of code in comments, README, and any documents or notebooks.
    
  \bibliographystyle{IEEEtran}
  \bibliography{bibliography}

\end{document}